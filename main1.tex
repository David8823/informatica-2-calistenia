\documentclass{article}
\usepackage[utf8]{inputenc}
\usepackage{graphicx}


\begin{document}
\begin{titlepage}
    \begin{center}
        \vspace*{1cm}
            
        \huge
        \textbf{Informatica 2 }
            
        \vspace{0.5cm}
        \LARGE
        Calistenia
            
        \vspace{1.5cm}
            
        \textbf{David Correa Ochoa}
            
        \vfill
            
        \vspace{0.8cm}
            
        \Large
        Despartamento de Ingeniería Electrónica y Telecomunicaciones\\
        Universidad de Antioquia\\
        Medellín\\
        Marzo de 2021
            
    \end{center}
\end{titlepage}

\begin{center}
    \huge
    Pasos para la actividad
\end{center}


\begin{enumerate}
    \vspace{0.8cm}
    \item levantar la hoja de papel y dejarla a un lado de las tarjetas a como estaba en su estado inicial
    \item agarrar la mano los bordes de las tarjetas  y reposarlas en el centro de la hoja de la misma forma de como estaban inicialmente
    \item agarrar de nuevo las tarjetas de manera que uno de los bordes mas cortos quede hacia arriba
    \item 
    \item en las tarjetas ubicar el pulgar en una de las esquinas superiores de las tarjetas  y hacer lo mismo con la esquina contraria superior con el dedo del medio 
    \item agarrar las 2 tarjetas con los 2 dedos antes mencionados en esas ubicaciones 
    \item levantar la parte superior de las tarjetas apoyando la parte inferior de las tarjetas en la hoja mientras sigue agarrando la parte superior hasta que queden totalmente     vertical
    \item cuando esten levantadas las tarjetas colocar el dedo indice en la parte superior de las tarjetas en mitad de los 2 dedos utilizados anteriormente
    \item mientras apoya la parte inferior de las tarjetas en la hoja, ubicar el dedo anular en el centro del borde de las tarjesta que este del lado del dedo anular
    \item hacer presion con el dedo anular en la ranura  que esta entre las 2 trajetas hasta separar la parte inferior entre las dos tarjetas 
    \item luego de separarlas apoyar el dedo anular en una de las tarjetas e ir empujandola hacia el lado contrario de la otra tarjeta 
    \item empujar con el dedo anular hasta que los bordes de las tarjetas formen un triangulo 
    \item al formar el triangulo soltar suavemente las tarjetas, si las tarjetas caen volver al paso 2 y separar un poco mas las tarjetdas
    \item si las tarjetas quedan equilibradas se ha finalizado la actividad con exito!
\end{enumerate}

\end{document}



















